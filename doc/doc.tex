\documentclass[a4paper,10pt]{article}
	 \textheight = 24cm
	 \textwidth = 18cm
	 \topmargin = -1cm
	 \oddsidemargin = -1cm
	 
\setcounter{part}{0} 	
\setcounter{secnumdepth}{3} % Para enumerar hasta subsubsection
\setcounter{tocdepth}{3} % Para enumerar hasta subsubsection 

\usepackage[utf8]{inputenc}
\usepackage[none]{hyphenat} %Paquete para indicar que no separe las palabras
\usepackage{graphicx} %Paquete para incluir imágenes
\graphicspath{ {images/} } %Indicamos la carpeta donde están las imágenes
\usepackage{fancyhdr} %Paquete para modificar los pies de página y encabezados
\usepackage[hidelinks]{hyperref} 
\usepackage{enumitem}
\usepackage{hyperref}
\usepackage{color}

\usepackage{listings} %Paquete para poner código de programación


\lstset{frame=tb,
  language=C,
  aboveskip=3mm,
  belowskip=3mm,
  showstringspaces=false,
  columns=flexible,
  basicstyle={\small\ttfamily},
  numbers=left,
  numberstyle=\tiny\color{gray},
  keywordstyle=\color{blue},
  commentstyle=\color{dkgreen},
  stringstyle=\color{mauve},
  breaklines=true,
  breakatwhitespace=true,
  tabsize=3
}

\renewcommand{\refname}{Bibliografía}
\renewcommand{\appendixname}{Apéndices}
\renewcommand{\contentsname}{Índice}
\renewcommand{\partname}{Parte}
\renewcommand{\thesection}{\arabic{section}}
\renewcommand{\figurename}{Figura}

\begin{document}
%--------------------------------------------------------------------
\thispagestyle{empty}
\begin{figure}[h]
\includegraphics[scale=0.4]{logo-umu.jpg} \hspace{100mm}
\includegraphics[scale=0.2]{logo-fium.png}
\centering
\end{figure}

\vspace{15mm}

\begin{center}
\rule{100mm}{0.1mm} \\
\vspace{5mm}
\begin{Huge}
 \textbf{Universidad de Murcia}\\ \vspace{10mm}
\end{Huge}
\begin{huge}
 \textbf{Facultad de Informática}
\end{huge}
\vspace{5mm} \\
\rule{100mm}{0.5mm}

\vspace{20mm}

\begin{Huge}
 IA para el Desarrollo de Videojuegos \\ \vspace{5mm}
\end{Huge}

\begin{LARGE}
 Real Time Wargame
\end{LARGE}

\vspace{15mm}

\begin{Large}
 \textbf{Autores} \\ \vspace{2mm}
 Antonio López Martínez-Carrasco\\ \vspace{2mm}
 \textit{antonio.lopez31@um.es} \\ \vspace{4mm}
 José María Sánchez Salas\\ \vspace{2mm}
 \textit{josemaria.sanchez12@um.es}
\end{Large}

\vspace{7mm}

\begin{Large}
 \textbf{Profesores} \\ \vspace{2mm}
 Francisco Javier Marín-Blázquez Gómez \\ \vspace{2mm}
 \textit{jgmarin@um.es} \\ \vspace{4mm}
 Luis Daniel Hernández Molinero \\ \vspace{2mm}
 \textit{ldaniel@um.es}
\end{Large}
\end{center}
%--------------------------------------------------------------------
%--------------------------------------------------------------------
\newpage
\pagestyle{fancy}
\renewcommand{\headrulewidth}{0.5pt} %Definimos el grosor de la línea
\lhead[IA para el Desarrollo de Videojuegos]{IA para el Desarrollo de Videojuegos}
\rhead[Real Time Wargame]{Real Time Wargame}
\renewcommand{\footrulewidth}{0.5pt}
\tableofcontents
\newpage

\newpage
\part{Introducción y estructura de la aplicación}
%--------------------------------------------------------------------
\medskip
\section{Introducción}
Este documento corresponde con la documentación del proyecto Real Time Wargame propuesto por la asignatura IA para el Desarrollo de Videojuegos perteneciente a la mención de Computación del 4º curso del Grado de Ingeniería Informática impartida por la Universidad de Murcia. \\

El proyecto consiste en implementar algunos elementos de inteligencia artificial en un entorno de juego de guerra en tiempo real. Estos elementos de inteligencia artificial van desde la parte reactiva, como comportamientos más básicos, como puede ser moverse de un sitio a otro, hasta la parte táctica, como comportamientos tácticos basados en una máquina de estados. \\

Para la implementación de este proyecto, hemos decidido utilizar la librería gráfica LibGDX \cite{libgdx}. El motivo por el cuál nos hemos decantado por esta librería es porque es una librería sin licencia, implementada en Java, con muy buena documentación y que no requiere de un hardware específico para su ejecución, además de que estamos muy acostumbrados a trabajar en Java. Así pues, no hemos implementado este proyecto para ningún hardware específico, simplemente hemos hecho uso del IDE Eclipse junto con la librería LibGDX, programando y ejecutándolo en nuestros respectivos ordenadores. \\

Este documento consta de ocho partes fundamentales: la primera de ellas se encarga de introducir el proyecto, así como también la estructura que tiene el proyecto desarrollado en Java. La segunda contiene un breve manual de uso para los usuarios. La tercera contiene la explicación de la clase principal del programa, el mapa implementado y la interacción con el usuario. La cuarta parte comenta toda la parte de la inteligencia artificial reactiva que hemos implementado, así como el modelo de objetos del videojuego que hemos desarrollado. La quinta trata las diversas implementaciones del Pathfinding que hemos llevado a cabo. La sexta parte contiene la parte de la inteligencia artificial táctica que hemos implementado, así como las modificaciones que hemos llevado a cabo en el modelo mencionado en la sección cuarta para poder introducir la parte táctica al modelo. La séptima parte comenta la implementación que se ha llevado a cabo del Flocking pedido en el proyecto. Y por último, en la parte octava, se hace una recopilación de todos los elementos opcionales que hemos implementado. \\

Las dos últimas partes de este documento constan de las conclusiones que hemos obtenido tras la realización de este proyecto y la bibliografía consultada para el mismo.



%--------------------------------------------------------------------
\medskip
\section{Estructura de la aplicación}


\newpage
\part{Manual de uso}
%--------------------------------------------------------------------
\medskip
\section{Manual de uso}


\newpage
\part{Clase principal, mapa e interacción con el usuario}
%--------------------------------------------------------------------
\medskip
\section{Clase principal}
Todo lo referente a esta sección, se encuentra dentro del paquete \texttt{com.mygdx.iadevproject} del proyecto. La clase principal de nuestro proyecto es la clase \texttt{IADeVProject} y es la que contiene los siguientes elementos:
\begin{itemize}
 \item \textbf{Constantes}. Estas son: el tamaño del mapa (alto y ancho); el tamaño de las celdas de los grids, así como el tamaño de estos (alto y ancho); valores de infinito, coste y terreno por defecto; y el tamaño de los objetos del mundo (alto y ancho, obtenidos a partir de la información del mapa).
 \item \textbf{Mapas}. Estos son: el mapa real que se dibuja; el mapa de costes para el PathFinding; el mapa de terrenos para modificar la velocidad de un personaje dependiendo del terreno por donde se mueve (exceptuando el mapa real, todos los demás son los que se denominan grids y todos tienen el mismo tamaño).
 \item \textbf{Variables globales}. Estas son: los objetos y obstáculos del mundo; el conjunto de objetos seleccionados por el usuario (utilizando el ratón); la cámara; el flag que indica si se dibujan, o no, las líneas de depuración de los distintos comportamientos; así como las variables necesarias para que desde los comportamientos se puedan dibujar las líneas de depuración. También están las bases y manantiales de los equipos, el controlador de la interacción con el usuario, las variables que indican si se pausa (o no) el juego y si se dibuja (o no) el mapa de influencia encima del mapa real, así como las variables que indican qué equipo ha ganado (si lo hay) y la variable que indica si puede haber (o no) ganador en la partida.
\end{itemize}

Esta clase se encarga de crear e inicializar todos los elementos anteriores, de renderizar el seguimiento del juego y, una vez terminado, eliminar todos los elementos creados para el renderizado. Para la creación de todos los personajes, se ha creado la clase \texttt{CreateCharacters} que proporciona un único método estático \texttt{createCharacters()}. También se encarga de inicializar los waypoints del juego, así como los mapas de influencia. Para esto, hace uso de los métodos estáticos proporcionados por las clases \texttt{Waypoints} y \texttt{SimpleMapOfInfluence} (véase secciones \ref{waypoints} y \ref{mapas-influencia}).\\

El control de la interacción del usuario con el juego se ha implementado en la clase \texttt{InputProcessorIADeVProject} (ver sección \ref{interaccion}), que contiene todos los manejadores para las posibles acciones que puede realizar el usuario con el teclado y el ratón. Debido a que el usuario puede seleccionar objetos del mundo, cuando se realiza la selección para que añadan los objetos al conjunto de objetos seleccionados, se hace uso del método \texttt{addToSelectedObjectsList()} de la clase \texttt{IADeVProject}. Como también se proporciona la funcionalidad de eliminar la selección de objetos, dicha clase también proporciona el método \texttt{clearSelectedObjectsList()}. \\

Debido a que el tamaño de los distintos grids y el tamaño del mapa pueden no ser iguales (pues depende del tamaño de las celdas), es necesario hacer una conversión entre las posiciones reales del mapa y las posiciones correspondientes del grid. Por eso mismo, la clase \texttt{IADeVProject} proporciona los dos siguientes métodos:
\begin{itemize}
 \item \texttt{mapPositionTOgridPosition() y gridPositionTOmapPosition()}. Ambos reciben como parámetros el tamaño de la celda del grid y un vector con la posición que se quiere convertir. El primero transforma una posición del mapa a una posición del grid (índices de la matriz que representa al grid), y el segundo hace la inversa: transforma una posición del grid a una posición del mapa.
\end{itemize}

También proporciona métodos para obtener las bases y manantiales de cada equipo, así como la posición de ambas.



\newpage
%--------------------------------------------------------------------
\medskip
\section{Mapa}
Todo lo referente a esta sección, se encuentra dentro del paquete \texttt{com.mygdx.iadevproject.map} del proyecto. El mapa que se ha diseñado para este proyecto se muestra en la Figura \ref{mapa:mapa}.
\begin{figure}[!th]
\includegraphics[scale=0.6]{map}
\centering
\caption{Mapa diseñado para el proyecto.}
\label{mapa:mapa}
\end{figure}

Como se puede observar, existen seis tipos de terrenos:
\begin{itemize}
 \item \textbf{Agua y montañas}. Representan los terrenos infranqueables. Como indica el enunciado del proyecto, los dos países se encuentran separados por un río con cuatro puentes para cruzarlo.
 \item \textbf{Desierto y bosque}. Se identifican de manera directa en el mapa. También se tiene como bosque los árboles que se encuentran dentro la pradera situada en la parte inferior izquierda del mapa.
 \item \textbf{Pradera}. Se corresponde con los tramos de color verde más claro.
 \item \textbf{Sendero}. Se corresponde con el fondo del mapa.
 \item \textbf{Camino}. Se corresponde con los tramos de color gris. Las bases de los países se encuentran encima de este tipo de terreno.
\end{itemize}

Para representar los terrenos en el proyecto, se ha implementado el enumerado \texttt{Ground} que proporciona los siguientes métodos:
\begin{itemize}
 \item \texttt{getCost()}. Debido a que cada terreno tiene un coste asociado, este método devuelve dicho coste.
 \item \texttt{getGround(int cost)}. Se trata de un método estático que dado un coste, devuelve el terreno al que pertenece. Si el coste que se pasa como argumento no se corresponde con ningún terreno, devuelve \texttt{null}.
\end{itemize}

También podemos observar que cada uno de los equipos tiene un manantial donde los personajes pueden curarse y a donde van cuando estos mueren. \\

Para diseñar el mapa, se ha hecho uso de la herramienta \texttt{Tiled Map Editor} \cite{tiledMap} que proporciona una sencilla forma de diseñar mapas por capas, permitiendo que puedas crear capas de distintos tipos (de tiles, de objetos, etc), lo que nos facilita la creación e inicialización de los distintos grids que utilizamos. Lógicamente, las capas que se encuentran en el mapa, se corresponden con los distintos tipos de terrenos mencionados anteriormente (excluyendo la capa de objetos). \\

Para introducir el mapa diseñado en el proyecto, \texttt{LibGDX} nos proporciona la clase \texttt{TiledMap}. Sin embargo, esta clase, a la hora de renderizar, si existe alguna capa de objetos por defecto, no la dibuja. Por este motivo, implementamos la clase \texttt{TiledMapIADeVProject}, que sobreescribe los métodos de renderizado para que se dibujen las capas de objetos. \\

Una vez que tenemos el mapa diseñado e introducido en nuestro proyecto, hay que inicializar los distintos grids en correspondencia con dicho mapa. Para ello, está la clase \texttt{MapsCreatorIADeVProject}, que proporciona un único método estático \texttt{createMaps()} y que se encarga de recorrer, para cada terreno, su capa correspondiente e inicializar los grids a los valores correspondientes.




























%--------------------------------------------------------------------
\medskip
\section{Interacción con el usuario}
\label{interaccion}
Todo lo referente a esta sección, se encuentra dentro del paquete \texttt{com.mygdx.iadevproject.userInteraction} del proyecto. Para la interacción del sistema con el usuario, se ha hecho uso de la interfaz \texttt{InputProcessor} proporcionada por la librería LibGDX. Esta interfaz proporciona método que se llaman cada vez que, por ejemplo, se pulsa una tecla o se clicka con el ratón en la pantalla. \\

La implementación de la interacción se ha hecho siguiendo una máquina de estados. Debido a que el usuario puede realizar ciertas acciones con los personajes seleccionados, para cada tipo de acción, se ha creado un estado y dentro de ese estado, dependiendo de la tecla pulsada, se hará una acción u otra. El por qué se ha realizado de esta manera es porque hay ciertos comportamientos que requieren que el usuario seleccione a otro personaje, en una situación normal, el programa se quedaría esperando a que el usuario seleccione el personaje. Sin embargo, si se hace eso, el juego se paraliza completamente, incluso el manejo de eventos de la librería, por lo que quedar a la espera de que el usuario seleccione un personaje, no es factible. \\

Con una máquina de estados, sí podemos controlar que, cuando el usuario haya realizado la acción que se espera, se pase a otro estado que la aplique, sin tener que parar la ejecución del programa. \\

Para encapsular los comportamientos y acciones que el usuario pueda hacer, se ha creado la clase \texttt{UserInteraction} que es la encargada de proporcionar con métodos estáticos dichas acciones. También se encarga de mostrar los mensajes con las posibles acciones que puede realizar el usuario.


\newpage
\part{IA reactiva}
%--------------------------------------------------------------------
\medskip
\section{Steerings}

En el ámbito de los videojuegos, un sistema steering (o sistema de dirección, es español) es un mecanismo que \textit{propone movimientos} a los agentes involucrados en el videojuego en base al entorno local que les rodea, es decir, usando la información del mundo que los agentes captan a través de sus sentidos. Los steerings concretos que se han estudiado y que, por tanto, han sido implementados es este proyecto son: \textbf{Steering Uniforme} y \textbf{Steering Uniformemente Acelerado}. \\

Un Steering Uniforme solamente maneja velocidad lineal y velocidad angular, puesto que se da por hecho que en el movimiento no va a intervenir ningún tipo de aceleración (en decir, la aceleración es 0). A partir de dicha información contenida en el steering, el agente modificará su posición y orientación. Por el contrario, un Steering Uniformemente Acelerado sí contiene una aceleración lineal y una aceleración angular, puesto que en este tipo de movimientos sí intervienen aceleraciones (aceleración constante). A partir de esta información contenida en el steering, el agente modificará su velocidad lineal y angular. Al modificar la velocidad lineal y angular, la posición y orientación del agente también será modificada en consecuencia. 

\subsection{Interfaz Steering}

Durante el desarrollo del proyecto, nos ha parecido conveniente el poder manejar todos los steerings de manera uniforme independientemente de la clase de steering concreto. Esto ha sido posible gracias a ciertas características del lenguaje Java como son las clases abstractas, la herencia o la ligadura dinámica. Teniendo esto en cuenta, hemos implementado una clase abstracta vacía (clase \texttt{Steering}), tal que todos los steerings concretos heredarán de ella.

\subsection{Steering Uniforme}

Este tipo de steering ha sido implementado en la clase \texttt{Steering\_NoAcceleratedUnifMov} y, tal y como se ha comentado, hereda de la clase abstracta \texttt{Steering}. En esta clase podemos encontrar dos atributos llamados \texttt{velocity} (de tipo \texttt{Vector3}) y \texttt{rotation} (de tipo float). El primero hace referencia a la velocidad lineal (el vector velocidad) y el segundo hace referencia a la velocidad angular (un escalar). En esta clase también están presentes los métodos \textit{get} y \textit{set} correspondientes a ambos atributos. Además de estos métodos, también hay otro método denominado \texttt{getSpeed}, que devuelve el módulo del vector \texttt{velocity}. \\

Es muy importante remarcar que, es este ámbito, \textit{velocity} y \textit{speed} no significan lo mismo. El primero hace referencia al \textbf{vector} velocidad, mientras que el segundo hace referencia al \textbf{módulo} de dicho vector (es un escalar).

\subsection{Steering Uniformemente Acelerado}

Este tipo de steering ha sido implementado en la clase \texttt{Steering\_AcceleratedUnifMov} y, tal y como se ha comentado, hereda de la clase abstracta \texttt{Steering}. Esta clase tiene dos atributos llamados \texttt{lineal} (de tipo \texttt{Vector3}) y \texttt{angular} (de tipo \texttt{float}). El primero corresponde con la aceleración lineal (un vector) y el segundo con la aceleración angular (un escalar). También se han implementado los métodos \textit{get} y \textit{set} correspondientes a ambos atributos.

%--------------------------------------------------------------------
\medskip
\section{Comportamientos}
Todo lo referente a esta sección, se encuentra dentro del paquete \texttt{com.mygdx.iadevproject.aiReactive.behaviour} del proyecto.


%--------------------------------------------------------------------
\medskip
\subsection{Comportamientos no acelerados}
Todo lo referente a esta sección, se encuentra dentro del paquete \texttt{com.mygdx.iadevproject.aiReactive.behaviour.noAcceleratedUnifMov} del proyecto.



%--------------------------------------------------------------------
\medskip
\subsection{Comportamientos acelerados}
Todo lo referente a esta sección, se encuentra dentro del paquete \texttt{com.mygdx.iadevproject.aiReactive.behaviour.acceleratedUnifMov} del proyecto.



%--------------------------------------------------------------------
\medskip
\subsection{Comportamientos delegados}
Todo lo referente a esta sección, se encuentra dentro del paquete \texttt{com.mygdx.iadevproject.aiReactive.behaviour.delegated} del proyecto.



%--------------------------------------------------------------------
\medskip
\subsection{Comportamientos de grupo}
Todo lo referente a esta sección, se encuentra dentro del paquete \texttt{com.mygdx.iadevproject.aiReactive.behaviour.group} del proyecto.


%--------------------------------------------------------------------
\medskip
\section{Árbitros}
Todo lo referente a esta sección, se encuentra dentro del paquete \texttt{com.mygdx.iadevproject.aiReactive.arbitrator} del proyecto. Para generalizar el uso de los árbitros, se ha decidido crear la interfaz \texttt{Arbitrator} que tienen que implementar todos los distintos tipos de árbitros. Esta interfaz proporciona un solo método:
\begin{itemize}
 \item \texttt{getSteering()}, que recibe como parámetro un objeto del tipo \texttt{Map<Float, Behaviour>} que contiene el conjunto de comportamientos (con sus valores de importancia correspondientes) del que se quiere obtener un steering final. 
\end{itemize}

Dependiendo del tipo de árbitro que utilicemos, este método devolverá un steering u otro. Los distintos tipos de árbitros que se han implementado son los siguientes:
\begin{itemize}
 \item \textbf{Árbitro de mezcla ponderada}. Este tipo de árbitro, lo que hace es obtener un steering final como resultado de la mezcla de todos los steerings obtenidos por el conjunto de comportamientos, de manera ponderada. Es decir, para cada comportamiento, se obtiene su steering y el resultado del mismo se añade al steering final multiplicado por el valor de importancia asociado al comportamiento (de ahí que se reciba un objeto de tipo \texttt{Map<Float, Behaviour>} que para cada comportamiento se tiene su valor de importancia asociado). Debido a que hay dos tipos de steerings, como se ha comentado en la sección \ref{steerings}, se han implementado dos tipos de árbitros de mezcla ponderada: uno para los comportamientos que devuelve steerings acelerados (clase \texttt{WeightedBlendArbitrator\_Accelerated}) y otro para los comportamientos que devuelven steerings no acelerados (clase \texttt{WeightedBlendArbitrator\_NoAccelerated}). El funcionamiento es el mismo en ambos casos, solamente que si un comportamiento devuelve un steering del otro tipo, no se tiene en cuenta. 
 
 Ambos árbitros, en su constructor, reciben como parámetros la máxima aceleración y la máxima rotación (en el caso del acelerado), y la máxima velocidad y máxima orientación (en el caso del no acelerado) que dicho árbitro tiene permitido devolver. Por lo que, una vez que se ha calculado el steering final, se comprueba que el valor del steering no supere ninguno de los valores anteriores. 
 
 
 \item \textbf{Árbitro de prioridad}. Este tipo de árbitro considera que el conjunto de comportamientos recibido como parámetro se encuentra ordenado por prioridad, es decir: el primer objeto del conjunto es el que tiene mayor prioridad. Así pues, recorre dicho conjunto, obteniendo para el comportamiento actual su steering y si este steering es válido (es decir, es distinto de \texttt{null} y supera un determinado valor \texttt{epsilon}), el árbitro directamente devuelve este steering y termina. Si resulta que de todo el conjunto de comportamientos, ninguno es válido porque no ha superado el valor \texttt{epsilon}, el árbitro devuelve, en este caso, el steering obtenido por el último comportamiento del conjunto, independientemente del valor que tenga. 
 
 A diferencia del árbitro anterior, con este árbitro se puede usar comportamientos que devuelvan tanto steerings acelerados como no acelerados; lo que implica que el árbitro puede devolver cualquier tipo de steering. El determinado valor \texttt{epsilon} es un valor que se le pasa al árbitro en su constructor, y que indica el valor mínimo que un steering debe de tener para que se considere válido.
\end{itemize}









%--------------------------------------------------------------------
\medskip
\section{Modelo}

Esta es una de las partes más importantes del videojuego, puesto que en ella podemos encontrar a los \textbf{agentes} que intervendrán e interactuarán en el transcurso del juego. Concretamente, en este paquete podemos encontrar la implementación de los objetos del mundo (concepto general que lo engloba todo), la implementación de los personajes, la implementación de los obstáculos y la implementación de las formaciones (que han sido tratadas como un tipo especial de personaje).

\subsection{Clase WorldObject}

La clase \texttt{WorldObject} es un concepto general que representa a una entidad del juego. Todas las entidades concretas (descritas en los siguientes subapartados) heredarán de esta \textbf{clase abstracta}. Una cosa muy importante a tener en cuenta es que esta clase abstracta hereda a su vez de la clase \texttt{Sprite} de la librería libgdx. Esto se ha hecho así para aprovechar todas las características y propiedades que ya posee la clase \texttt{Sprite}, como por ejemplo, la posición, la orientación, la anchura, la altura, las funciones de dibujo u otra funcionalidad ya preprogramada en la librería. \\

Solamente hay que tener en cuenta una cosa muy importante: a lo que nosotros llamamos orientación, la librería lo llama rotación. Esto a su vez puede llevar a confusiones con la velocidad angular (que para nosotros es la rotación). Para evitar todos estos problemas de nomenclatura y todas las confusiones que esto pueda provocar, se han tomado ciertas medidas:

\begin{itemize}
	\item[-] Se han creado el par de funciones \texttt{getOrientation} y \texttt{setOrientation} que llaman a las funciones correspondiente del padre (\texttt{super.getRotation} y \texttt{super.setRotation}).
	\item[-] Se han sobreescrito las funciones \texttt{getRotation} y \texttt{setRotation} heredadas del padre para que si se llaman en el hijo (en la clase \texttt{WorldObject}), lancen una excepción. Estas funciones no se deben llamar nunca, puesto que si deseamos consultar o modificar la orientación de una entidad, deberemos hacer uso de las funciones \texttt{getOrientation} y \texttt{setOrientation} de la clase \texttt{WorldObject}.
\end{itemize}

Estas modificaciones solucionan el problema y evitan que se puedan producir confusiones con los nombres y errores de concepto a lo largo del resto del proyecto. \\

En cuanto a los atributos propios de la clase \texttt{WorldObject} (los no heredados de la clase \texttt{Sprite}), podemos encontrar la velocidad del personaje (vector velocidad de tipo \texttt{Vector3}), la velocidad angular del personaje (escalar de tipo float), el atributo \texttt{minBoxLength} o longitud mínima de la caja de detección () y la velocidad máxima del personaje (escalar de tipo float que hace referencia a la máxima velocidad que puede tener el personaje independientemente de su comportamiento). Este último atributo se ve reflejado en el método \texttt{setVelocity}. Si el módulo del vector que se pasa como parámetro supera a la velocidad máxima del personaje permitida, entonces el vector que se asigna al personaje es un vector con la misma dirección y sentido que el pasado como parámetro, pero con un módulo igual al atributo \texttt{maxSpeed} (máxima velocidad del personaje permitida). \\
 

\subsection{Clase Character}

\subsection{Clase Obstacle}

\subsection{Formaciones}


\newpage
\part{PathFinding}
%--------------------------------------------------------------------
\medskip
\section{PathFinding}

Tras haber implementado todos los comportamientos y tipos de movimientos que se pueden aplicar a los personajes de nuestro videojuego, es también muy importante contar con un mecanismo que nos permita obtener el camino adecuado para llegar desde un origen a un destino en función de distintos criterios y teniendo en cuenta cierta información del entorno. Dicho mecanismo se denomina \textit{PathFinding} y es un elemento muy importante en aquellos videojuegos en los que sea necesario (aunque, como también se ha visto en clase, otros videojuegos no lo necesitan, por lo que no lo implementan). \\

El mecanismo de PathFinding visto en clase se apoya fundamentalmente en 2 estructuras. Por un lado, es totalmente imprescindible la matriz de costes de terrero. Esta estructura es un grid ficticio que se sitúa sobre el mapa físico y en el que se almacenan los costes del terreno del mapa del juego. Por otro lado, también se necesita una matriz de distancias o heurística. En esta matriz se almacenan los costes desde cada una de las celdas hasta una celda objetivo. Ambas matrices deben tener las mismas dimensiones. \\

Es muy importantes destacar la existencia de 2 tipos de posiciones con las que vamos a trabajar. Por un lado, tenemos la posición real de una entidad en el mapa y, por otro lado, dicha entidad también tendrá una posición en el grid (esta posición indica \textbf{los índices} de la matriz que representa al grid). Esta segunda posición es con la que trabaja el PathFinding. Teniendo esto en cuenta, habrá que contar con las funciones necesarias para pasar de un tipo de posición a otro. \\

A la hora de pasar una posición del mapa a una posición del grid, entre otras cosas, se eliminan los decimales para poder obtener una posición entera (para usar como índice en el grid ficticio). Por tanto, se podría decir que el mecanismo de búsqueda está basado en un \textbf{grid cuadrado por vértices}. Sin embargo, cuando volvemos a convertir de una posición del grid a una posición del mapa, nos interesa que el movimiento se realice por el centro del tile para evitar que el personaje se mueva justo por la separación entre un tipo de terreno y otro (justo entre un tile y otro). Por tanto, una de las cosas que se hace al transformar de una posición del grid a una posición del mapa es \textbf{sumar la mitad del lado de un tile a ambas coordenadas de la posición final}. De esta forma conseguimos que los puntos finales del mapa siempre estén en el centro de los tiles que representan el terreno. \\

Para este proyecto, hemos implementado \textbf{2 tipos de Pathfinding}. Ambos se basan en el algoritmo LRTA* con espacio de búsqueda minimal. La diferencia fundamental entre ambos es que uno calcula todos los punto de golpe desde el origen al destino (pathfinding continuo) y el otro va devolviendo punto por punto y va calculando un nuevo punto cada vez que sea necesario (pathfinding punto a punto). El primero de ellos devuelve una lista completa con todos los puntos desde el origen hasta el destino y basta con ejecutarlo tan solo una vez. El segundo de ellos devuelve un único punto (aunque también en una lista) y debe ser ejecutado cada vez que queramos obtener el siguiente punto al que ir. Esto se explicará más detalladamente en el apartado correspondiente.

\subsection{Distancias}

Tal y como se ha comentado en clase, las heurísticas que se han implementado son: \textbf{distancia de Manhattan}, \textbf{distancia de Chebyshev} y \textbf{distancia Euclídea}. La información concreta y la forma de cálculo de cada una de estas distancias se especifica detalladamente en las transparencia de la asignatura. \\

En le código fuente de nuestro proyecto podemos encontrar la interfaz \texttt{Distance}. Esta interfaz sirve para generalizar todos los tipos de distancia concretos y tratarlos de manera homogénea. El método fundamental de todas las distancias implementadas es \texttt{getMatrixOfDistances}. Este método recibe las dimensiones que deberá tener la matriz de salida y la posición objetivo y devuelve la matriz de distancias correspondiente. \\

A la hora de realizar un PathFinding desde un origen hasta un destino, la posición objetivo que se le pasa al método \texttt{getMatrixOfDistances} es la posición destino a la que queremos llegar. Esto quiere decir que la matriz de distancias calculada almacenará la distancia desde todas las posiciones del grid a la posición destino a la que queremos llegar (posición destino \textbf{del grid}).

\subsection{Algoritmo LRTA*}

Tal y como se dice en la especificación de estas prácticas, se debe implementar el algoritmo LRTA* con espacio de búsqueda minimal. Eso quiere decir que el espacio de búsqueda solamente estará compuesto por el estado actual en el que nos encontramos. \\

Para el caso del pathfinding continuo, el algoritmo LRTA* recibe como entrada la matriz de costes del terreno, la matriz de distancias (que será distinta según la heurística elegida), la anchura y altura de ambas matrices, la posición origen \textbf{del grid} y la posición destino \textbf{del grid}. Para el caso del pathfinding punto a punto no se recibirá la posición inicial, sino que cada vez que lo llamemos le pasaremos la posición actual desde la que queremos obtener el siguiente punto (que, efectivamente, al principio será el punto de origen). El algoritmo LRTA* trabajará (modificará) sobre la matriz de distancias y devolverá una lista de puntos que corresponden con todas aquellas posiciones \textbf{del grid} por las que hay que pasar para llegar desde el origen hasta el destino (para el caso de pathfinding continuo) o, directamente, el siguiente punto al que debemos ir (para el caso de pathfinding punto a punto). En este segundo caso, obviamente, el algoritmo no realiza ningún bucle. \\ 

Para calcular la función \textit{f(x)} de un estado concreto, serán necesarios los siguientes elementos:
\begin{itemize}
	\item[-] El coste que supone realizar una acción (movernos de una celda a otra). Este valor viene definido por la constante \texttt{default\_action\_cost} de la clase \texttt{LRTA\_star}.
	\item[-] La distancia que hay desde en estado actual hasta la posición de destino. Este valor podemos obtenerlo de la matriz de distancias.
	\item[-] El coste del terreno en la posición o estado actual en el que nos encontramos.
\end{itemize}

Estos componentes serán la base para obtener el valor de la función \textit{f(x)} de un estado concreto y, por tanto, para que el algoritmo LRTA* calcule el camino adecuado desde un origen hasta un destino. \\

Además, también se han implementado diversos método necesarios para el correcto funcionamiento del algoritmo LRTA*. El primero de ellos es \texttt{generateSuccessors}. Este método recibe una posición del grid y devuelve todos sus vecinos (tanto en horizontal, como en vertical, como en diagonal). El segundo método necesario es \texttt{getSuccessorWithTheSmallestHeuristic}. Este método selecciona el vecino tal que \textit{f(x)} devuelva el menor valor.

\subsection{Clase PathFinding}

Esta es la clase principal del mecanismo de PathFinding. Al igual que ocurre con el propio algoritmo LRTA*, cada tipo de pathfinding implementado tiene su propia clase y lo implementa de distinta forma. Las particularidades de cada uno se comentarán en las siguientes secciones.

\subsubsection{Pathfinding continuo}

En este pathfinding el método \texttt{applyPathFinding} es el encargado de realizar la conversión de posiciones del mapa a posiciones del grid (y viceversa) y de la ejecución del algoritmo LRTA*. En este caso, la función \texttt{applyPathFinding} devolverá el resultado final (lista con todos los puntos) tras la primera ejecución y, si queremos volver a hacer un pathfinding, deberemos crear y comenzar de nuevo. Este método recibe como parámetro la matriz de costes del terreno, la heurística deseada, el tamaño del lado de celda del grid, la anchura y altura de las matrices, las coordenadas 'x' e 'y' \textbf{del mapa} del origen y las coordenadas 'x' e 'y' \textbf{del mapa} del destino. \\

En primer lugar, las coordenadas del mapa son transformadas en coordenadas del grid; seguidamente, se calcula la matriz de distancias; a continuación, se ejecuta el algoritmo LRTA*; y finalmente, se vuelven a transformar todas las coordenadas del grid al mapa (de todos los puntos que ha devuelto el algoritmo LRTA*). \\

Cabe destacar que al transformar las coordenadas del mapa de los puntos origen y destino a coordenadas del grid se pierde información, puesto que las coordenadas del mapa son de tipo \texttt{float} (con decimales) y las coordenadas del grid no tienen decimales (puesto que estas coordenadas son realmente los indices con los que se accederán a ambas matrices). Por este motivo, al volver a transformar las coordenadas del grid al mapa, se realiza también una segunda transformación. El primer y último punto de la lista que devuelve el algoritmo LRTA* (el origen y destino), son reemplazados por las coordenadas originales que se pasaron como parámetro a la función \texttt{applyPathFinding}.

\subsubsection{Pathfinding punto a punto}

Al contrario que pasaba con el pathfinding anterior, en este caso el constructor de la clase juega un papel más importante. En este caso, es en el constructor donde se almacenan todos los atributos necesarios (según los parámetros pasados al constructor) y donde se realiza la transformación de la coordenada inicial del pathfinding de coordenadas del mapa a coordenadas del grid. Además, puesto que en esta ocasión el pathfinding va a ser llamado más de una vez (cosa que no pasaba en el pathfinding anterior), en el constructor también se crea el objeto que representa el algoritmo LRTA* y se almacena como un atributo de la clase (para ir usando siempre el mismo objeto y no perder el avance realizado). \\

Un elemento fundamental en este tipo de pathfinding es el atributo \texttt{objetivoActual}. Inicialmente, este atributo contiene el punto de partida del pathfinding y, conforme vayamos avanzando, iremos actualizando este atributo con la posición a la que debemos ir. Este caso, el pathfinding punto a punto irá devolviendo siempre la misma posición hasta que no lleguemos a ella (ver las comprobaciones iniciales del método \texttt{applyPathFinding}). Una vez que el personaje haya alcanzado el siguiente objetivo actual, el pathfinging y el algoritmo LRTA* continuarán calculando.

\subsection{Pathfinding táctico}

Para añadir información táctica al pathfinding es suficiente con tener en cuenta el valor de la matriz de influencia (o cualquier otra información que queramos añadir) en el cálculo de la función f(x) en el algoritmo LRTA*. La utilización de información táctica se puede activar o desactivar sobre la marcha. Para ello, hemos añadido los flags correspondientes en las clases del pathfinding y del algoritmo LRTA* y las funciones necesarias para activar y desactivar dichos flags. \\

Concretamente, nuestro pathfinding táctico podrá usar la información táctica proveniente del mapa de influencia y la información táctica propia de cada rol. Dicha información táctica en relación a cada rol concreto se puede encontrar en el método \texttt{getTacticalCost} de las clases \texttt{Archer} y \texttt{Soldier}. Cada rol concreto podrá tener un coste diferente según cada uno de los terrenos presentes en el mapa.



\newpage
\part{IA táctica}

%--------------------------------------------------------------------
\medskip
\section{Modificaciones en el modelo}
Para incluir la parte táctica hemos tenido que añadir cierta funcionalidad y atributos en la clase \texttt{Character}, así como crear un enumerado que refleje los equipos posibles en el videojuego. 

%--------------------------------------------------------------------
\medskip
\subsection{Clase Character}
Las modificaciones que se han hecho dentro de la clase \texttt{Character} explicada en la sección \ref{clase-personaje} han sido las siguientes:
\begin{itemize}
 \item Se ha añadido el atributo estático \texttt{DEFAULT\_HEALTH} que refleja la vida por defecto que tienen todos los personajes. Este atributo está inicializado a 20.000.
 \item Se ha añadido un atributo de tipo \texttt{TacticalRole} (ver sección \ref{roles}) que contiene el rol del personaje.
 \item Se ha añadido un atributo de tipo \texttt{Team} que contiene el equipo al que pertenece el personaje.
 \item Se han añadido dos atributos para controlar la vida del personaje: uno para la vida actual y otra para la vida máxima que puede tener un personaje.
 \item Debido a que puede haber interacción con el usuario y que si un personaje es seleccionado, este debe hacer caso a lo que indica el usuario, se ha añadido un atributo booleano que indica si el rol de un personaje está habilitado o no.
 \item Debido a que ahora es el rol del personaje el que le indica qué comportamientos tiene que realizar, la aplicación de los comportamientos sigue siendo la misma, pero en vez de llamar directamente al método \texttt{applyBehaviour()}, previamente hay que actualizar el rol para que actualice la lista de comportamientos del personaje (ver sección \ref{roles} donde se explican los roles). Para esto, se ha creado el método \texttt{updateTacticalRole()} que sencillamente lo que hace es, si el rol está activo (no ha sido seleccionado por el usuario) actualizarlo (siempre y cuando el personaje no pertenezca a una formación) y después llamar al método \texttt{applyBehaviour()} tal y como se explica en la sección \ref{clase-personaje}.
 \item Un personaje, por defecto, no tiene por qué tener un rol. Para asignarle un rol, se ha creado el método \texttt{initializeTacticalRole()} que recibe como parámetro un objeto del tipo \texttt{TacticalRole} y que lo que hace es asignar el rol al personaje e inicializarlo.
 \item Para que, dependiendo del rol del personaje, este se mueva más o menos rápido dependiendo del terreno que esté pisando, se ha creado el método \texttt{getVelocityFactorOfThisCharacter()} que devuelve un valor entre 0 y 1, que indica cómo se aplica la velocidad obtenida del steering obtenido al personaje (método \texttt{update()}). Si el personaje no tiene rol o está en una formación que no tiene rol, este método siempre devolverá un 1, es decir, no se verá afectada la velocidad del personaje con respecto al terreno que esté pisando. Si el personaje no tiene rol, pero pertenece a una formación con rol, entonces, este método devolverá lo que devuelva el método \texttt{getVelocityFactor()} del rol de la formación. Sin embargo, si tiene rol, este método devolverá lo que devuelva el método \texttt{getVelocityFactor()} del rol (ver sección \ref{roles}).
 \item También se han creado dos métodos para reducir y aumentar la vida del personaje, que se usarán en los comportamientos de ataque y curación. Estos métodos reciben como parámetro la vida que se quiere reducir/aumentar. La vida del personaje se visualiza siempre y cuando este no pertenezca a una formación (en este caso, se visualizará solamente la vida de la formación), para ello, también se ha creado otro método que dibuja la vida del personaje al lado del mismo.
 \item Como se ha comentado, que el usuario seleccione a un personaje, implica que su rol se deshabilite y el personaje haga solamente lo que el usuario le mande. De igual manera, cuando el usuario lo deseleccione, el personaje debe volver a aplicar su rol como si no hubiera pasado nada. Para ello, se han creado dos métodos, uno para indicar que el personaje ha sido seleccionado y otro para indicar que el personaje ha sido deseleccionado.
 \item Por último, para que se vayan mostrando el estado táctico por el que pasa el personaje, también se ha implementado un método que recibe como parámetro una cadena con el estado en el que está (además de los objetos necesarios para dibujar). Este método se llamará en todos los estados/nodos de los roles tácticos para ir indicando qué es lo que está realizando el personaje.
\end{itemize}



%--------------------------------------------------------------------
\medskip
\subsection{Enumerado Team}
Para poder reflejar los equipos del videojuego, se ha creado el enumerado \texttt{Team} que contiene tres equipos: \texttt{FJAVIER, LDANIEL y NEUTRAL}. Como se puede apreciar, los nombres de los equipos los hemos puesto en honor a los dos profesores de la asignatura. El equipo de \texttt{FJAVIER} es el que se sitúa en la base de abajo a la derecha, mientras que el equipo de \texttt{LDANIEL} es el que sitúa en la base de arriba a la izquierda. El equipo \texttt{NEUTRAL} es un equipo neutral y se añadió para poder poner otros personajes en el videojuego que no pertenezcan a ningún equipo (tal como animales o cosas por el estilo). Sin embargo, en el diseño final no hemos incluido personajes neutrales, pero así, ya está todo preparado por si se quieren introducir. \\

Este enumerado tiene un único método interesante: \texttt{getEnemyTeam()} que devuelve el equipo contrario al objeto que realiza el método: si el equipo es \texttt{LDANIEL} devolverá \texttt{FJAVIER} (y viceversa), pero si es \texttt{NEUTRAL} devolverá \texttt{NEUTRAL}.















%--------------------------------------------------------------------
\medskip
\section{Otros comportamientos}

A parte de todos los comportamientos descritos anteriormente, para la parte táctica también se han implementado algunos otros comportamientos. Realmente, no son comportamientos como tal, sino que son diversas acciones o procesos para los cuales ha resultado conveniente el poder tratarlos homogéneamente y como si fueran comportamientos normales. Estas acciones son el \textbf{ataque} y la \textbf{cura}. Cabe destacar que estos comportamientos \textbf{devolverán el steering nulo}, puesto que, realmente, lo importante de estos comportamientos no es el comportamiento propiamente dicho, sino todas las funciones/comprobaciones que se realizan dentro de él (en el método \texttt{getSteering}).

\subsection{Comportamiento de ataque}

Este comportamiento representa la orden o la acción de atacar a un objetivo. Por tanto, los elementos que este comportamiento necesita como entrada son: el personaje que realiza el ataque, el personaje que recibe el ataque, el daño que realiza el personaje origen al personaje destino (salud que se le resta al personaje destino) y la máxima distancia a la que se puede realizar el ataque. \\

En primer lugar, se comprueba si la distancia a la que se encuentran origen y destino es menor o igual que la distancia máxima permitida para realizar el ataque. En caso afirmativo, se llamará al método \texttt{reduceHealth} de la clase \texttt{Character}, para reducir la vida al target. \\

Cuando un personaje (no formación) realiza un ataque, este comportamiento será añadido a su lista de comportamiento. Tal y como se puede observar, el hecho que tratar el mecanismo de ataque como un comportamiento más, nos permite poder añadirlo a la lista de comportamientos de un personaje y poder tratarlo de manera homogénea con el resto de comportamientos.

\subsection{Comportamiento de cura}

Este comportamiento representa el proceso de cura (incremento de salud) por parte de un personaje. Por tanto, los elementos que este comportamiento necesita como entrada son: el personaje fuente cuya salud va a ser incrementada y el valor de salud a incrementar. \\

En este caso, lo único que se hace en el método \texttt{getSteering} es llamar al método \texttt{addHealth} de la clase \texttt{Character}. Con esto conseguimos incrementar el nivel de salud de un personaje fuente. \\

Al igual que antes, cuando un personaje (no formación) se cura, este comportamiento será añadido a su lista de comportamientos.

\subsection{Ataque y cura en las formaciones}

A la hora de entender los procesos de ataque y cura para el caso de las formaciones, hay que tener muy claro lo siguiente:
\begin{itemize}
	\item Cuando un conjunto de personajes están en una formación, la lista de comportamientos propia de cada personaje no se tiene en cuenta. Por el contrario, se usará la lista de comportamiento del propio objeto formación (para mover el ancla de la formación) y un conjunto de comportamientos establecidos directamente ``a pelo'' en el método correspondiente para hacer que los componentes de la formación vayan al punto que les corresponde.
	\item Los objetos formación como tal \textbf{ni atacan ni se curan}. Los que atacan son los elementos de la formación.
	\item Del mismo modo, los objetos de tipo formación no pueden ser atacados (a la hora de atacar se comprueba que el objetivo no sea una formación) ni se pueden curar (ya que, de nuevo, los que se curan son los componentes de la formación).
	\item Los componentes de la formación no tienen estructura táctica (máquina de estados o árbol de decisión). Esta estructura solamente se encuentra en el objeto tipo formación (en la ``raíz'', en caso de haber formaciones de formaciones).
\end{itemize}

Sabiendo todo esto, es necesario contar con algún mecanismo para el objeto tipo formación (el ancla) pueda comunicar a los integrantes de la formación que deben curarse o atacar a un objetivo determinado. Esta comunicación se realiza a través de los métodos \texttt{enableAttackMode} y \texttt{disableAttackMode} (para el caso del comportamiento de ataque) y los métodos \texttt{enableCure} y \texttt{disableCure} (estos métodos se encuentran en la clase \texttt{Formation}). Al llamar a estos métodos lo que hacemos es modificar los flags correspondientes y almacenar otros atributos necesarios para los comportamientos de ataque y cura. Teniendo en cuenta esos flags, en el método \texttt{getComponentFormationSteerginToApply} de cada tipo de formación concreta se añadirá el comportamiento de ataque o cura a la lista de comportamientos establecida ``a pelo'' que, como sabemos, se usará para controlar a cada uno de los componentes de la formación. De esta manera conseguimos el poder imponer el comportamiento de ataque o cura a cada uno de los elementos que componen una formación.



%--------------------------------------------------------------------
\medskip
\section{Acciones y comprobaciones}

% DOCUMENTAR SOLAMENTE PARA QUÉ LO HEMOS CREADO Y PONER ALGUNOS EJEMPLILLOS TONTITOS:
% PA MÁS INFORMACIÓN, QUE MIREN EN LA DOCUMENTACIÓN DE JAVADOC


%--------------------------------------------------------------------
\medskip
\section{Roles Tácticos}
\label{roles}
La implementación de los roles tácticos la hemos realizado a través de la interfaz \texttt{TacticalRole} que contiene dos métodos importantes:
\begin{itemize}
 \item Método \texttt{initialize()}. Este método realiza la inicialización de la estructura táctica que se va a utilizar. Por ejemplo, en el caso de que un rol se implemente con una máquina de estados, pues crear e inicializar dicha máquina.
 \item Método \texttt{update()}. Este método realiza la actualización de la estructura táctica inicializada y las acciones pertinentes dependiendo de dicha actualización.
\end{itemize}

Ambos métodos reciben como parámetro el personaje al que se va a aplicar los comportamientos obtenidos tras la inicialización/actualización de la estructura táctica. \\

También contiene los siguientes métodos:
\begin{itemize}
 \item \texttt{getVelocityFactor()}. Devuelve el factor de velocidad que un rol tiene para un determinado terreno. Este método devolverá un valor entre 0 y 1 que se usa para modificar la velocidad que se aplica a un personaje antes de aplicar un determinado steering. Con este método, se permite que los personajes, dependiendo de su rol, vayan a una velocidad dependiendo del terreno por el que vayan.
 \item \texttt{getTacticalCost()}. Devuelve el coste táctico que un rol tiene asociado a un determinado terreno. 
 \item \texttt{getMaxDistanceOfAttack()}. Devuelve la máxima distancia de ataque de un rol.
 \item \texttt{getDamageToDone()}. Devuelve el daño que puede hacer al atacar un rol.
 \item \texttt{getMaxSpeed()}. Devuelve la máxima velocidad a la que puede ir un rol.
\end{itemize}

Los roles tácticos que hemos implementado nosotros han sido: soldado y arquero, tanto ofensivos como defensivos. Tanto los ofensivos como los defensivos, tienen los mismos valores para los métodos anteriores (dependiendo de si son arqueros o soldados). La diferencia entre ellos está en la estructura con la que se han implementado: los roles ofensivos (tanto soldados como arqueros) se han implementado con un árbol de decisión; mientras que los roles defensivos (tanto soldados como arqueros) se han implementado con una máquina de estados. En la siguientes subseciones, se comentará más en profundidad sobre cada uno de estos roles. \\

Todo esto se encuentra dentro del paquete \texttt{com.mygdx.iadevproject.aiTactical.roles} del proyecto.


%--------------------------------------------------------------------
\medskip
\subsection{Roles defensivos}
Ambos roles defensivos (arquero y soldado) se han implementado como una máquina de estados. Para ello, se ha hecho uso de la interfaz \texttt{StateMachine} proporcionada por la librería LibGDX \cite{stateMachine}. Esta máquina de estados hace uso de la interfaz \texttt{State} proporcionada también por la librería LibGDX, que proporciona los siguientes métodos:
\begin{itemize}
  \item \texttt{enter()} que se llama cada vez que se entra al estado. 
  \item \texttt{update()} que se llama cada vez que la máquina de estados se actualiza y este es el estado actual de la máquina.
  \item \texttt{exit()} que se llama cuando se sale del estado.
  \item \texttt{onMessage()} que se llama si la entidad recibe un mensaje del despachador de mensajes mientras está en este estado. 
\end{itemize}

Estos métodos reciben como parámetro una entidad con la que se puede trabajar en cada método. En nuestro caso, esta entidad será un objeto de la clase \texttt{Character}. \\

La interfaz \texttt{StateMachine} proporciona varios métodos para poder realizar distintas acciones con la máquina de estados. De entre ellos, las que hemos utilizado son:
\begin{itemize}
 \item \texttt{setInitialState()}. Se emplea para establecer el estado inicial de la máquina, que se le pasa como parámetro.
 \item \texttt{isInState()}. Comprueba si la máquina está en el estado pasado como parámetro.
 \item \texttt{changeState()}. Cambia el estado actual al estado pasado como parámetro.
 \item \texttt{update()}. Actualiza la máquina: esto implica llamar al método \texttt{update()} del estado actual. 
\end{itemize}

Al crear la máquina de estados, esta pide que se indique el objeto ``propietario'' de la máquina de estados; esto es, el objeto que la máquina pasa como parámetro en los métodos de la interfaz \texttt{State}. Para introducirle los estados que nosotros queremos a la máquina de estados, hemos tenido que crear clases concretas que implementaran la interfaz \texttt{State} anteriormente mencionada. Para todos los estados, la entidad que recibe como parámetro es de la clase \texttt{Character}. \\

Por último, en las Figuras \ref{defensivos:soldado} y \ref{defensivos:arquero} se muestran las máquinas de estados concretas para los soldados y arqueros defensivos. La idea que hay detrás de estos roles es que ambos tienen que patrullar una zona (los soldados patrullan su base, los arqueros patrullan los waypoints de los puentes), hasta que se encuentran a un enemigo cerca y lo atacan, siempre y cuando no estén lejos de su base/waypoint. Como se trata de un rol defensivo, estos van a estar atacando hasta que se mueran. Por último, debido a que hay 6 waypoints para cada patrullar, si hay más arqueros defensivos que waypoints, estos se quedarán a la espera (haciendo cosas aleatorias) de que uno se libere para cogerlo.
\begin{figure}[!th]
\includegraphics[scale=0.6]{defensive-soldier}
\centering
\caption{Máquina de estados para el soldado defensivo.}
\label{defensivos:soldado}
\end{figure}
\begin{figure}[!th]
\includegraphics[scale=0.6]{defensive-archer}
\centering
\caption{Máquina de estados para el arquero defensivo.}
\label{defensivos:arquero}
\end{figure}

Es importante destacar los siguientes puntos:
\begin{itemize}
 \item Si no se cumplen las condiciones necesarias en el estado actual, no se cambia de estado. De ahí que no haya transiciones en las figuras sobre los propios estados. Cuando no se cambia de estado, exceptuando el estado \texttt{AttackEnemies} y los que requieren el uso del Pathfinding, la máquina de estados no se actualiza, ya que cuando se entró al estado, se calcularon los comportamientos correspondientes y el personaje sigue haciendo lo mismo, por lo que no es necesario calcularlo de nuevo.
 
 \item Los estados que requieran el uso del Pathfinding (\texttt{GoToMyBase, GoToMyManantial y GoToMyWaypoint}) para evitar que se esté calculando el Pathfinding cada vez que se actualice el estado, hacemos uso del método \texttt{enter()} del estado para que cada vez que se entre se calcule el Pathfinding, mientras que en el método \texttt{update()} lo que se hace es aplicar el Pathfinding para que nos dé el siguiente punto a seguir.
 
 \item En el estado \texttt{AttackEnemies} se ataca siempre al enemigo más cercano (de ahí que se tenga que actualizar el estado si no hemos cambiado). También el personaje intenta alejarse del objetivo a una distancia en la que él pueda atacar. Así pues permitimos que aquellos personajes que tengan una distancia de ataque mayor (los arqueros en nuestro caso), puedan sacar provecho de ello cuando se ataca. También se mira al personaje que se ataca.
 
 \item En el estado \texttt{IAmDead} se traslada directamente el personaje a la posición de su manantial, estableciendo su posición a la posición del manantial.
 
 \item En el estado \texttt{BookWaypoint} mientras que espera a tener un waypoint libre, el personaje lo que hace es moverse de manera aleatoria (aplicando un Wander), para evitar que se quede parado.
 
 \item Debido a que los arqueros defensivos tienen que reservar el waypoint que tienen que patrullar, solamente se va a liberar un waypoint cuando el arquero que lo está patrullando muere. Es decir, entrar al estado \texttt{IAmDead} implica que el arquero deja libre su waypoint para que otro pueda cogerlo. Sin embargo, cuando un arquero entra al estado \texttt{GoToMyManantial} porque quiere curarse, no se libera el waypoint porque no ha muerto. 
 
 \item Consideramos que de primeras, los soldados se encuentran en la base (o cerca de ella), por lo que en su estado inicial no se hace ningún Pathfinding. En cambio, como los arqueros, mientras que no tienen un waypoint que patrullar, se mueven de manera aleatoria, cuando reserva un waypoint, sí se calcula un Pathfinding para ir a él, porque puede estar en la otra punta del mapa.
\end{itemize}


%--------------------------------------------------------------------
\medskip
\subsection{Roles ofensivos}


%--------------------------------------------------------------------
\medskip
\section{Waypoints}
\label{waypoints}

%--------------------------------------------------------------------
\medskip
\section{Puntos de moral}

%--------------------------------------------------------------------
\medskip
\section{Mapas de influencia}

Los mapas de influencia reflejan la presencia o \textit{poder} de cada uno de los equipos en una zona o casilla concreta del mapa. Para cada personaje del equipo, la influencia del ese equipo comenzará desde la casilla actual en la que se encuentra el personaje y podrá extenderse una determinada distancia (una cantidad determinada de casillas alrededor del personaje). Aunque, es habitual que conforme mayor sea la distancia menor sea la influencia que ese personaje va teniendo sobre el terreno. \\

Tal y como hemos planteado los mapas de influencia, cada equipo tendrá su propio mapa de influencia y el contenido de cada uno de esos mapas se calculará en función de la posición de los personajes de ese equipo (solamente los de ese equipo). Esta separación inicial en diversos mapas nos servirá para disponer y controlar de una forma más exhaustiva la presencia y valores concretos de cada uno de los personajes de cada equipo. Cuando vayamos a dibujar la influencia final sobre el terreno, sí que tendrán en cuenta todos los mapas de todos los equipos. Es importante tener en cuenta que, aunque cada equipo tenga su propio mapa, \textbf{el equipo con mayor influencia en una casilla, controlará dicha casilla} (esto se tendrá en cuenta al dibujar el mapa). \\

A la hora de calcular la influencia que ejerce un personaje concreto sobre el terreno que le rodea, son necesarios 2 valores: el máximo valor de influencia (que será la influencia que se ejercerá sobre la casilla en la que se encuentra el personaje) y la máxima distancia de influencia (a partir de esa distancia, el personaje no ejercerá influencia sobre el terreno). Conforme nos vamos alejando de la casilla actual en la que se encuentra el personaje, \textbf{el valor de influencia va disminuyendo en 1}. El mecanismo de cálculo es la distancia de Chebyshev (tal y como se ha visto en la parte de pathfinding, aunque con algunas modificaciones). Teniendo en cuenta cual es el resultado de esta distancia, hemos considerado que aplicarla en este caso es la mejor opción y lo más conveniente. A la hora de modificar con la matriz de influencia de un equipo, solo se tendrá en cuenta la región afectada por un personaje concreto (de acuerdo a la distancia máxima establecida), es decir, que para cada personaje no se recorrerá la matriz entera. Cabe destacar que un personaje neutral no ejercerá influencia, no tendrá ninguna matriz asociada y no podrá tener acceso a esta funcionalidad. \\

Cuando hay varios personajes del mismo equipo cerca, a la hora de calcular el mapa de influencia de ese equipo, puede ocurrir que en valor de una celda supere el valor máximo permitido (ya que las influencias individuales se van sumando). Cuando esto ocurre simplemente sustituimos dicho valor por el máximo permitido. \\ 

Esta funcionalidad consta básicamente de los siguientes métodos:
\begin{itemize}
	\item En primer lugar, deberemos inicializar todas las variable necesarias usando los métodos habilitados para ello. Esto solo e hace 1 vez.
	\item A continuación, en cada iteración del bucle del juego debemos llamar al método \texttt{updateSimpleMapOfInfluence} para actualizar la información de los mapas de influencia. Cabe destacar que solamente se tendrán en cuenta los personajes como tal y no los objetos de tipo formación.
	\item Finalmente, debemos llamar al método de dibujo para poder visualizar la influencia que ejerce cada equipo sobre el terreno. O bien en forma de mini mapa o bien sobre el propio mapa completo (en función de los parámetro que pasemos a la función).	
\end{itemize}

Para realizar el dibujo del mapa de influencia calculado, existe el método \texttt{drawInfluenceMap}. Este método recibe los siguientes parámetros:
\begin{itemize}
	\item renderer.
	\item output\_grid\_cell\_size $\rightarrow$ Tamaño del lado de cada cuadradito o celda del mapa de influencia.
	\item positionX y positionY $\rightarrow$ Posición de la esquina inferior izquierda del mapa de influencia.
	\item filled $\rightarrow$ Booleano que indica si las celdas del mapa de influencia se rellenarán o no. Si este flag es falso, solamente se colorearán los lados de las celdas.
\end{itemize}

Ajustando el tamaño de celda y las posiciones de mapa podemos conseguir que el mapa de influencia se muestra como un minimapa junto al mapa real o sobre el propio mapa real (con los cuadrados no rellenos, en este caso). \\

A la hora de dibujar el mapa, cada equipo tiene distintos tonos de azul y distintos tonos de rojo (un equipo es el color azul y otro es el rojo). Además, el color banco representará una casilla neutral. Para cada posición del mapa de influencia, a parte de comprobar si la celda debe rellenarse/colorearse o no, también se harán otras comprobaciones:
\begin{itemize}
	\item Si el mapa de costes de terreno en esa posición corresponde con un terreno infranqueable, no se dibujará la influencia.
	\item Si la influencia de un equipo es mayor a la del otro, se dibujará del color correspondiente al equipo con mayor influencia y del tono de color correspondiente. Para obtener el todo de color correspondiente \textbf{se resta el valor del equipo con mayor influencia menos el valor del equipo con menor influencia} (puesto que aunque un equipo controle una casilla, la influencia del otro equipo también afecta). Ese resultado se divide entre 5 (porque hay 5 tonos de color por cada equipo) y se usa el tono correspondiente al cociente obtenido.
	\item Finalmente, si el parámetro \texttt{filled} está a true. Se dibuja la celda de color blanco. Esto se usa para que, en el caso del minimapa, sí aparenza el fondo blanco, pero en el caso de dibujar el mapa de influencia sobre el mapa real, solamente aparezcan las influencias de los equipos y no se dibujen las zonas sin influencia.
\end{itemize}

A la hora de integran esta información con el pathfinding táctico de un personaje, habría que pasar el mapa de influencia \textbf{del equipo contrario}. El pathfinding debe tener en cuenta la influencia del equipo rival (coste añadido) para NO IR por esas zonas. Por tanto, a priori no tiene sentido usar el mapa de influencia de mi propio equipo, ya que estaría penalizando las zonas controladas por mí. \\

Para finalizar, voy a adjuntar una captura de pantalla de nuestro videojuego en la que se puede observar el mapa de influencia final dibujado (tanto en forma de minimapa como sobre el mapa real):
\begin{figure}[!th]
\includegraphics[scale=0.6]{influencia}
\centering
\caption{Mapa de influencia.}
\label{mapa:mapa}
\end{figure}



\newpage
\part{Flocking}

%--------------------------------------------------------------------
\medskip
\section{Flocking}
En esta sección va a tratar sobre la implementación del Flocking pedido en el proyecto. Todo lo referente a esta sección se encuentra dentro del paquete \texttt{com.mygdx.iadevproject.aiReactive.flocking}, en la carpeta de test. \\

Debido a que la finalidad del videojuego no es muy idílica para incluir un flocking en él, hemos decidido implementar el flocking a parte (de ahí que esté en el apartado de test, y no en la parte de código fuente). \\

Para conseguir este comportamiento, hemos utilizado un árbitro por mezcla ponderada y donde todos los personajes tienen los siguientes comportamientos:
\begin{itemize}
 \item \texttt{CollisionAvoidance} para evitar choques.
 \item \texttt{Separation} para separase del grupo.
 \item \texttt{Cohesion} para juntarse al grupo.
 \item \texttt{VelocityMatching} para ajustarse a la velocidad del personaje que guía al flocking.
 \item \texttt{LookingWhereYouGoing} para que miren por dónde van.
 \item \texttt{Wander} para que parezca que hacen cosas aleatorias.
\end{itemize}

Cada uno de estos comportamientos tienen un peso distinto, por ejemplo, el \texttt{CollisionAvoidance} es el que tiene mayor peso, mientras que el \texttt{Wander} es el que tiene menor peso. Para guiar a todo el grupo, se ha creado un personaje que simplemente realiza un \texttt{Arrive} a la posición que se clicka en pantalla. Todos los demás personajes hacen el \texttt{VelocityMatching} a este personaje. \\

El funcionamiento del test es el siguiente: hay cinco personajes (cubos) que son a los que se aplica el Flocking y otro personaje (gota) que es al que el grupo de cubos hace el \texttt{VelocityMatching}. De esta manera, el usuario puede clickar en la pantalla y la gota lo que hará será ir hacia esa posición realizando un \texttt{Arrive}, lo cual hará que los cubos empiecen a ir en esa dirección. Hay que tener en cuenta que la gota \textbf{simplemente sirve para modificar la dirección de la velocidad del grupo}; es decir, \textbf{el Flocking no se aplica a la gota}.



\newpage
\part{Elementos opcionales}

%--------------------------------------------------------------------
\medskip
\section{Elementos opcionales}

En esta sección se van a explicar todos los elementos opcionales que también han sido implementados e incluidos en nuestro proyecto. \\

El primero de ellos ha sido la implementación de distintos comportamientos básico, delegados y en grupo. De hecho, hemos implementado prácticamente casi todos los comportamientos que hemos estudiado y que aparecen en las transparencias de la asignatura. La información sobre todos los comportamientos implementados se podrá encontrar en el apartado correspondiente. \\

Del mismo modo, también hemos implementado e incluido algunas estructuras de arbitraje como los \textbf{árbitros por prioridad} y los \textbf{árbitros por mezcla/por pesos} (tanto para los comportamientos acelerados como para los no acelerados). Se podrá encontrar una documentación más detallada en el apartado correspondiente. \\

Para el caso de las formaciones, se han implementado algunas otras estructuras a parte de la propia formación en circulo. Estos nuevos tipos de formación son: \textbf{formación en línea} y \textbf{formación en estrella}. Además de esto, también hemos añadido la posibilidad de crear \textbf{formaciones de formaciones} todo lo profundas que deseemos (mediante en patrón de diseño \textit{composite}). Esto se explica más detalladamente en el apartado correspondiente. \\

Uno de los elementos opcionales más importantes que hemos hecho ha sido el \textbf{modo depuración}. Durante una partida y pulsando las teclas adecuadas, se puede visualizar un montón de información sobre los movimientos que realizan los personajes, el estado de los personajes o el comportamiento táctico de los mismos. Para ello, en prácticamente todos los comportamientos implementados hemos añadido un método llamado \texttt{debug}, que será el encargado de dibujar toda la información de depuración necesaria para ese comportamiento. Además, en la clase \texttt{Character} también se han implementado ciertos métodos para mostrar, por ejemplo, el estado táctico en el que se encuentra el personaje. Por otro lado, en las formaciones también se dibujan los puntos a los deben ir los componentes de la propia formación. Para el caso de los waypoints, también contamos con los métodos necesarios para dibujar dichos puntos sobre el mapa del juego. \\

Tal y como se acaba de comentar, hemos implementado información de depuración para prácticamente todos los comportamientos (entre los que se incluye el Pathfollowing). Por tanto, como este comportamiento se usa para ir a los puntos devueltos por el pathfinding, también se podría decir que ha sido implementado el modo de depuración para el pathfinding, donde se muestran todos aquellos puntos que han sido obtenidos y por donde el personaje deberá ir. \\

Debido a cómo se han planteado las formaciones, será el propio objeto formación (el ancla) el que ejecute el pathfinding y el que vaya por los puntos que éste devuelva (los componentes de la formación simplemente se limitarán a ir a los puntos que les diga el ancla). Por tanto, para el caso de las formaciones, el pathfinding (normal y táctico) se lleva a cabo en un nivel superior y el camino obtenido se le impone a los componentes de la formación (ya que como acabo de decir, será en ancla el que lo ejecute y no cada uno de los componentes de la formación). \\

En cuanto a la información táctica implementada, no solo se ha hecho la parte de los mapas de influencia, sino que también se ha añadido un coste táctico a cada uno de los roles implementados en este proyecto (distinto coste para cada tipo de terreno). Ese coste táctico podrá repercutir directamente en la ejecución del pathfindind táctico (siempre y cuando, el personaje que lo ejecuta tenga un rol). \\

Tal y como comentaba anteriormente, debido a cómo se han planteado las formaciones, será el propio objeto formación (el ancla) el que tome ciertas decisiones y los componentes simplemente se limitarán a ir al punto que se les indique (y a acatar las decisiones tomadas por el ancla). Por tanto, ese mismo principio repercute directamente en las acciones de ataque o curación. Para el caso de las formaciones, la decisión de atacar (y también de curarse) no es tomada por los propios componentes de la formación, sino que se toma en una nivel superior (la toma el objeto tipo formación o ancla y se la impone a sus componentes).






\newpage
\part{Conclusiones}
%--------------------------------------------------------------------
\medskip
\section{Conclusiones}

Este proyecto ha consistido en implementar una serie de elementos de inteligencia artificial que se usan muy comúnmente en los videojuegos reales. Además, la propia implementación de dichos elementos se ha llevado a cabo en el contexto de un juego de guerra en tiempo real. Tal y como se ha podido apreciar a lo largo de la documentación, esos elementos van desde la parte reactiva (los comportamientos más básicos, los movimientos de un punto origen a un punto destino y las estructuras que permiten elegir o combinar el comportamiento o comportamientos a aplicar) hasta la parte táctica (que se ha abordado mediante ciertas estructuras de comportamiento táctico y toma de decisiones, como máquina de estados o árboles de comportamiento), pasando además por otras técnicas interesantes que también se usan hoy día en muchos videojuegos reales (mapas de influencia o waypoints). Así mismo, también se ha diseñado e implementado otra funcionalidad para permitir agrupar a los personajes y gestionarlos de manera compacta (formaciones de personajes). \\

El estudio, aplicación e implementación de estas técnicas nos ha permitido comprender de una forma más profunda y mejor el funcionamiento interno de los videojuego reales y qué está pasando realmente dentro de ellos cuando vemos determinadas acciones y comportamientos en una partida. Así mismo, también hemos podido reforzar y coger experiencia con algunos conceptos y técnicas muy interesantes que, muy posiblemente, puedan tener una aplicación práctica en otros ámbitos y no solamente en el mundo de los videojuegos. A parte de los anterior, los conceptos estudiados en la asignatura y la realización de este proyecto también nos ha permitido estudiar y aprender la estructura general de un videojuego, cómo éste se organiza y cómo las distintas partes que lo componen interactúan y se comunican entre sí. \\

Por todo lo anterior, consideramos que estas prácticas han sido muy adecuadas e interesantes, nos han permitido introducirnos (al menos, de forma básica) en el mundo de los videojuegos y en la implementación de éstos y nos han permitido comprender y entender la estructura y organización general de un videojuego.




\newpage
\part{Bibliografía}
\begin{thebibliography}{aaaa}
\bibitem{libgdx} \textsc{Badlogic Games}, \textit{LibGDX} \href{https://libgdx.badlogicgames.com/}{\color{blue}\underline{Enlace}} (última visita 16/05/2017).

\bibitem{stateMachine} \textsc{LibGDX}, \textit{State Machine.} \href{https://github.com/libgdx/gdx-ai/wiki/State-Machine}{\color{blue}\underline{Enlace}} (última visita 15/05/2017).

\bibitem{apuntes} \textsc{Profesores de la Asignatura}, \textit{Apuntes teóricos.} 2017. AulaVirtual. 

\bibitem{tiledMap} \textsc{Thorbørn Lindeijer et al.}, \textit{Tiled Map Editor.} \href{http://www.mapeditor.org/}{\color{blue}\underline{Enlace}} (última visita 06/05/2017).
\end{thebibliography}

\end{document}