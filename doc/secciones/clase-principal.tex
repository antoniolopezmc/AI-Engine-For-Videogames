%--------------------------------------------------------------------
\medskip
\section{Clase principal}
Todo lo referente a esta sección, se encuentra dentro del paquete \texttt{com.mygdx.iadevproject} del proyecto. La clase principal de nuestro proyecto es la clase \texttt{IADeVProject} y es la que contiene los siguientes elementos:
\begin{itemize}
 \item \textbf{Constantes}. Estas son: el tamaño del mapa (alto y ancho); el tamaño de las celdas de los grids, así como el tamaño de estos (alto y ancho); valores de infinito, coste y terreno por defecto; y el tamaño de los objetos del mundo (alto y ancho, obtenidos a partir de la información del mapa).
 \item \textbf{Mapas}. Estos son: el mapa real que se dibuja; el mapa de costes para el PathFinding; el mapa de terrenos para modificar la velocidad de un personaje dependiendo del terreno por donde se mueve (exceptuando el mapa real, todos los demás son los que se denominan grids y todos tienen el mismo tamaño).
 \item \textbf{Variables globales}. Estas son: los objetos y obstáculos del mundo; el conjunto de objetos seleccionados por el usuario (utilizando el ratón); la cámara; el flag que indica si se dibujan, o no, las líneas de depuración de los distintos comportamientos; así como las variables necesarias para que desde los comportamientos se puedan dibujar las líneas de depuración. También están las bases y manantiales de los equipos, el controlador de la interacción con el usuario, las variables que indican si se pausa (o no) el juego y si se dibuja (o no) el mapa de influencia encima del mapa real, así como las variables que indican qué equipo ha ganado (si lo hay) y la variable que indica si puede haber (o no) ganador en la partida.
\end{itemize}

Esta clase se encarga de crear e inicializar todos los elementos anteriores, de renderizar el seguimiento del juego y, una vez terminado, eliminar todos los elementos creados para el renderizado. Para la creación de todos los personajes, se ha creado la clase \texttt{CreateCharacters} que proporciona un único método estático \texttt{createCharacters()}. También se encarga de inicializar los waypoints del juego, así como los mapas de influencia. Para esto, hace uso de los métodos estáticos proporcionados por las clases \texttt{Waypoints} y \texttt{SimpleMapOfInfluence} (véase secciones \ref{waypoints} y \ref{mapas-influencia}).\\

El control de la interacción del usuario con el juego se ha implementado en la clase \texttt{InputProcessorIADeVProject} (ver sección \ref{interaccion}), que contiene todos los manejadores para las posibles acciones que puede realizar el usuario con el teclado y el ratón. Debido a que el usuario puede seleccionar objetos del mundo, cuando se realiza la selección para que añadan los objetos al conjunto de objetos seleccionados, se hace uso del método \texttt{addToSelectedObjectsList()} de la clase \texttt{IADeVProject}. Como también se proporciona la funcionalidad de eliminar la selección de objetos, dicha clase también proporciona el método \texttt{clearSelectedObjectsList()}. \\

Debido a que el tamaño de los distintos grids y el tamaño del mapa pueden no ser iguales (pues depende del tamaño de las celdas), es necesario hacer una conversión entre las posiciones reales del mapa y las posiciones correspondientes del grid. Por eso mismo, la clase \texttt{IADeVProject} proporciona los dos siguientes métodos:
\begin{itemize}
 \item \texttt{mapPositionTOgridPosition() y gridPositionTOmapPosition()}. Ambos reciben como parámetros el tamaño de la celda del grid y un vector con la posición que se quiere convertir. El primero transforma una posición del mapa a una posición del grid (índices de la matriz que representa al grid), y el segundo hace la inversa: transforma una posición del grid a una posición del mapa.
\end{itemize}

También proporciona métodos para obtener las bases y manantiales de cada equipo, así como la posición de ambas.

