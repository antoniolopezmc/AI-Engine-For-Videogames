%--------------------------------------------------------------------
\medskip
\section{Introducción}
Este documento corresponde con la documentación del proyecto Real Time Wargame propuesto por la asignatura IA para el Desarrollo de Videojuegos perteneciente a la mención de Computación del 4º curso del Grado de Ingeniería Informática impartida por la Universidad de Murcia. \\

El proyecto consiste en implementar algunos elementos de inteligencia artificial en un entorno de juego de guerra en tiempo real. Estos elementos de inteligencia artificial van desde la parte reactiva, como comportamientos más básicos, como puede ser moverse de un sitio a otro, hasta la parte táctica, como comportamientos tácticos basados en una máquina de estados. \\

Para la implementación de este proyecto, hemos decidido utilizar la librería gráfica LibGDX \cite{libgdx}. El motivo por el cuál nos hemos decantado por esta librería es porque es una librería sin licencia, implementada en Java, con muy buena documentación y que no requiere de un hardware específico para su ejecución, además de que estamos muy acostumbrados a trabajar en Java. Así pues, no hemos implementado este proyecto para ningún hardware específico, simplemente hemos hecho uso del IDE Eclipse junto con la librería LibGDX, programando y ejecutándolo en nuestros respectivos ordenadores. \\

Este documento consta de ocho partes fundamentales: la primera de ellas se encarga de introducir el proyecto, así como también la estructura que tiene el proyecto desarrollado en Java. La segunda contiene un breve manual de uso para los usuarios. La tercera contiene la explicación de la clase principal del programa, el mapa implementado y la interacción con el usuario. La cuarta parte comenta toda la parte de la inteligencia artificial reactiva que hemos implementado, así como el modelo de objetos del videojuego que hemos desarrollado. La quinta trata las diversas implementaciones del Pathfinding que hemos llevado a cabo. La sexta parte contiene la parte de la inteligencia artificial táctica que hemos implementado, así como las modificaciones que hemos llevado a cabo en el modelo mencionado en la sección cuarta para poder introducir la parte táctica al modelo. La séptima parte comenta la implementación que se ha llevado a cabo del Flocking pedido en el proyecto. Y por último, en la parte octava, se hace una recopilación de todos los elementos opcionales que hemos implementado. \\

Las dos últimas partes de este documento constan de las conclusiones que hemos obtenido tras la realización de este proyecto y la bibliografía consultada para el mismo.

